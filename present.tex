\documentclass{beamer}
\usepackage{listings} % source code
\usepackage{color}
\usepackage{graphicx}
\usepackage{verbatim}

\usetheme{Dresden}

\title{Necko overview}
\author{Patrick Wang}

\begin{document}

\begin{frame}
  \titlepage
\end{frame}

\begin{frame}{Introduction}
  \begin{itemize}
  \item \textbf{Necko's primary responsibility} is moving data from one location,
    to another location. 
  \item Used to be a standalone module, until Gecko 9.0
  \end{itemize}
\end{frame}

\begin{frame}{Basic component: from a user's perspective}
  \begin{itemize}
  \item URI
    \begin{itemize}
    \item \textbf{URI} represents location
    \item \texttt{nsIURI}, \texttt{nsSimpleURI}, \ldots
    \item \texttt{nsINestedURI} standards for `a uri that contains another uri'
    \end{itemize}
  \item URL
    \begin{itemize}
    \item \textbf{A URL} implements URI interface.
    \item URLs provide getting/setting of paths, hosts, ports, filenames.
    \end{itemize}
  \item Channel
    \begin{itemize}
    \item \texttt{nsIChannel} provides a data access interface which allows you to read or write data from or to a URI
    \end{itemize}
  \end{itemize}
\end{frame}

\begin{frame}{Creating URI and Channel}
  \begin{itemize}
  \item Build \texttt{URI} and \texttt{Channel} though \texttt{nsIOService}
  \end{itemize}
  \begin{center}
    \includegraphics[height=0.7\textheight]{ns-io-service.pdf}
  \end{center}
\end{frame}

\begin{frame}[fragile]
  \frametitle{Add a new protocol}
  To add a new protocol, you will need to implement an \texttt{nsIProtocolHandler}.
\begin{verbatim}
  readonly attribute ACString scheme;
  readonly attribute long defaultPort;
  readonly attribute unsigned long protocolFlags;

  nsIURI newURI(in AUTF8String aSpec,
                in string aOriginCharset,
                in nsIURI aBaseURI);
  nsIChannel newChannel(in nsIURI aURI);
  boolean allowPort(in long port, in string scheme);
\end{verbatim}
\end{frame}

\begin{frame}{Add a new protocol}
  \begin{itemize}
  \item Make contract id of the new protocol handler \texttt{NS\_NETWORK\_PROTOCOL\_CONTRACTID\_PREFIX "my-scheme"}
    \begin{itemize}
    \item Note \texttt{\#define NS\_NETWORK\_PROTOCOL\_CONTRACTID\_PREFIX "@mozilla.org/network/protocol;1?name="}
    \end{itemize}
  \end{itemize}
\end{frame}

% ---- Inside necko


\end{document}

